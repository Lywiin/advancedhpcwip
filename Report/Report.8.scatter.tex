\documentclass{article}
\usepackage[utf8]{inputenc}

\title{Report.8.scatter}
\author{Remi}
\date{November 2018}

\begin{document}

\maketitle

\section{Explain how you implement the labwork}

The labwork works the following way:
\begin{enumerate}  
\item Create a struct HSV which contains 3 arrays of double
\item Initialize input and output image as usual
\item Launch a first kernel to convert the image from rgb to hsv
    \subitem Store input in local variable to save access to global memory
    \subitem Find the min and max value between RGB
    \subitem Calculate HSV from the formula and put back result into local variables
    \subitem Then put back the final result into the output
\item Launch a second kernel to convert the image from hsv to rgb
    \subitem Store input into local variable to save access to global memory
    \subitem Calculate d, hi, f, l, m, n from the inputs
    \subitem Calculate new RGB values from these depending of hi value
    \subitem Put back the result into the output
\item Finally, write back the output image
\end{enumerate}

\section{Explain and measure speedup, if you have performance optimizations}
As I said in the previous block I used the following optimization:
\begin{enumerate}  
\item - Store input variables to local variable to save global memory access
\end{enumerate}

Same as previous labwork, hard to know the speedup of these optimization since I wrote an optimized version first...

\end{document}
