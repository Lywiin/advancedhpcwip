\documentclass{article}
\usepackage[utf8]{inputenc}

\title{Report.10.kuwahara}
\author{Remi}
\date{November 2018}

\begin{document}

\maketitle

\section{Explain how you implement the labwork}

The labwork works the following way:
\begin{enumerate}  
\item Create a struct SD which contain a double with the value of a standard deviation and a uchar3 containing the mean of a RGB pixel
\item Launch rgb2hsv kernel to convert input image to HSV
\item Launch the kernel that will compute the minimum standard deviation for each pixel
    \subitem First, calculate the border of the window chosen to not overlap image size
    \subitem Initialize some local variables that will be usefull to calculate the mean of V, R, G, B or the number of pixel in the window
    \subitem Do a first double for loop on the grid to sum the value V, R, G, B of each pixel for each window
    \subitem Do a for loop to calculate the mean for each window
    \subitem Do another double for loop to sum values in preparation for the standard deviation
    \subitem Do a final for loop to calculate the standard deviation of each window
    \subitem Find the minimum standard deviation and return that struct
\item Launch a final kernel to apply the computed mean RGB to each pixel
\item Finally, write back the output image
\end{enumerate}

\section{Explain and measure speedup, if you have performance optimizations}
For this labwork, I used the following optimization:
\begin{enumerate}  
\item - Store input variables to local variable to save global memory access
\end{enumerate}

Same as previous labwork, hard to know the speedup of these optimization since I wrote an optimized version first...

\ 

However, I think my algorithm can be improve doing the following:
\begin{enumerate}  
\item - Get rid of the structure which is not that useful here
\item - Assign the means in the same kernel instead of starting a new one
\end{enumerate}

\end{document}
