\documentclass{article}
\usepackage[utf8]{inputenc}

\title{ Report.5.gaussian.blur}
\author{Remi}
\date{November 2018}

\begin{document}

\maketitle

\section{Explain how you implement the Gaussian Blur filter}

I followed the following steps:
\begin{enumerate}  
\item Initialize input and output image as usual
\item I created two kernels
    \subitem One without shared memory
    \subitem One with shared memory
\item The only difference is the following:
    \subitem In the first one, the gaussian blur matrix is stored in the kernel in each thread
    \subitem In the second one, the gaussian blur matrix is stored in a shared variable, so one by block
\item Then for each kernel I do a double for loop to calculate the mean of each pixel with the weights in the gaussian blur matrix
\item Finally, the new value of the output is set
\end{enumerate}

\section{Try experimenting with different 2D block size values}

I tried experimenting with the following sizes:
\begin{enumerate} 
\item 8x8: takes about 29ms
\item 16x16: takes about 25ms
\item 32x32: takes about 26ms
\item 64x64: doesn't work anymore
\end{enumerate}
The more efficient seems to be 16x16 block size

\section{Compare speed between non-shared and shared memory}
Surprisingly, the speedup is not noticeable in this example between non-shared and shared memory.

\end{document}
