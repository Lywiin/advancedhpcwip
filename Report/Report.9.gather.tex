\documentclass{article}
\usepackage[utf8]{inputenc}

\title{Report.9.gather}
\author{Remi}
\date{November 2018}

\begin{document}

\maketitle

\section{Explain how you implement the labwork}

The labwork works the following way:
\begin{enumerate}  
\item Create a struct Histo which contains one array of int if size 256
\item Launch grayscale kernel to convert input image to gray image
\item Launch a first kernel to create local histograms from a part of the image (1 thread per line in my case, so height numbers of histo)
    \subitem Create a local empty histogram
    \subitem For each pixel increment the corresponding index of the histogram
    \subitem Write back the local histo in the output array
\item Launch a second kernel in a while loop to gather those histograms together until all the values are put together
\item Launch a third kernel to calculate probability of given intensity J
\item Launch a fourth kernel to compute cdf for each pixel
\item Launch a fifth kernel to equalize each pixel with the calculated h
\item Finally, write back the output image
\end{enumerate}

\section{Explain and measure speedup, if you have performance optimizations}
For this labwork, I used the following optimization:
\begin{enumerate}  
\item - Store input variables to local variable to save global memory access
\end{enumerate}

Same as previous labwork, hard to know the speedup of these optimization since I wrote an optimized version first...

\end{document}