\documentclass{article}
\usepackage[utf8]{inputenc}

\title{Report.7.reduce}
\author{Remi}
\date{November 2018}

\begin{document}

\maketitle

\section{Explain how you implement the labwork}

The labwork works the following way:
\begin{enumerate}  
\item Initialize input and output image as usual
\item Use grayscale kernel we created earlier to transform our image into grayscale image
\item Launch minmax kernel a first time, which will calculate the min and max value within the image
    \subitem Init shared variable by precomputing first result on the current reduce
    \subitem Use uchar3 to store min values into x and max values into y variables
    \subitem Then compute the rest of the reduce in an optimized way
        \subsubitem - Store input variables to local variable to save global memory access
        \subsubitem - No branch diversion
        \subsubitem - No memory bank conflict
        \subsubitem - No idle thread by reducing the number of thread launched by 2
    \subitem Write back the first input or the shared variable to the output with block index as index
\item Do a while loop launching minmax kernel again but with half the number of block each time
\item When number of block is 1, launch the minmax kernel a last time with 1 block then write back the first output x as min and y as max
\item Launch a last kernel called stretch to map the values of each pixel from the original range to a range from 0 to 255
\item Finally, write back the output image
\end{enumerate}

\section{Explain and measure speedup, if you have performance optimizations}
As I said in the previous block I used the following optimization:
\begin{enumerate}  
\item - Store input variables to local variable to save global memory access
\item - No branch diversion
\item - No memory bank conflict
\item - No idle thread by reducing the number of thread launched by 2
\end{enumerate}

Hard to know the speedup of these optimization though, since I wrote an optimized version first...

\section{Try experimenting with different 2D block size values}

I worked only in 1D in this labwork since it was simpler to understand, but here are the results of my experimentations:
\begin{enumerate} 
\item 256: takes about 26ms
\item 512: takes about 26ms
\item 1024: takes about 25ms
\item 2048: doesn't work anymore
\end{enumerate}
Here the difference is not really noticeable.
\end{document}
